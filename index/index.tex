% Options for packages loaded elsewhere
\PassOptionsToPackage{unicode}{hyperref}
\PassOptionsToPackage{hyphens}{url}
%
\documentclass[
]{article}
\usepackage{amsmath,amssymb}
\usepackage{lmodern}
\usepackage{ifxetex,ifluatex}
\ifnum 0\ifxetex 1\fi\ifluatex 1\fi=0 % if pdftex
  \usepackage[T1]{fontenc}
  \usepackage[utf8]{inputenc}
  \usepackage{textcomp} % provide euro and other symbols
\else % if luatex or xetex
  \usepackage{unicode-math}
  \defaultfontfeatures{Scale=MatchLowercase}
  \defaultfontfeatures[\rmfamily]{Ligatures=TeX,Scale=1}
\fi
% Use upquote if available, for straight quotes in verbatim environments
\IfFileExists{upquote.sty}{\usepackage{upquote}}{}
\IfFileExists{microtype.sty}{% use microtype if available
  \usepackage[]{microtype}
  \UseMicrotypeSet[protrusion]{basicmath} % disable protrusion for tt fonts
}{}
\makeatletter
\@ifundefined{KOMAClassName}{% if non-KOMA class
  \IfFileExists{parskip.sty}{%
    \usepackage{parskip}
  }{% else
    \setlength{\parindent}{0pt}
    \setlength{\parskip}{6pt plus 2pt minus 1pt}}
}{% if KOMA class
  \KOMAoptions{parskip=half}}
\makeatother
\usepackage{xcolor}
\IfFileExists{xurl.sty}{\usepackage{xurl}}{} % add URL line breaks if available
\IfFileExists{bookmark.sty}{\usepackage{bookmark}}{\usepackage{hyperref}}
\hypersetup{
  pdftitle={ERModPoi Method Discription},
  hidelinks,
  pdfcreator={LaTeX via pandoc}}
\urlstyle{same} % disable monospaced font for URLs
\usepackage[margin=1in]{geometry}
\usepackage{graphicx}
\makeatletter
\def\maxwidth{\ifdim\Gin@nat@width>\linewidth\linewidth\else\Gin@nat@width\fi}
\def\maxheight{\ifdim\Gin@nat@height>\textheight\textheight\else\Gin@nat@height\fi}
\makeatother
% Scale images if necessary, so that they will not overflow the page
% margins by default, and it is still possible to overwrite the defaults
% using explicit options in \includegraphics[width, height, ...]{}
\setkeys{Gin}{width=\maxwidth,height=\maxheight,keepaspectratio}
% Set default figure placement to htbp
\makeatletter
\def\fps@figure{htbp}
\makeatother
\setlength{\emergencystretch}{3em} % prevent overfull lines
\providecommand{\tightlist}{%
  \setlength{\itemsep}{0pt}\setlength{\parskip}{0pt}}
\setcounter{secnumdepth}{-\maxdimen} % remove section numbering
\ifluatex
  \usepackage{selnolig}  % disable illegal ligatures
\fi

\title{ERModPoi Method Discription}
\author{}
\date{\vspace{-2.5em}}

\begin{document}
\maketitle

\#METHODS

\#\#Modeling: Software and Strategy The E-R analyses, data processing,
and generation of figures and tables were performed in R.The poisson
regression analyses were performed using the glm() function in the R
programming language.

The analyses for each endpoint followed the below steps:

\begin{verbatim}
-Univariate models were estimated to compare and select the best exposure metric, if any. The model only contains exposure metric is the base model.
-Predicted cumulative incidence from the base model are created in form of figures and tables. 
-If covarariates are evaluated, then a full covariate model is estimated with the selected best exposure, if any, along with all covariates specified by the user.
-If covariates are to be evaluated, then a backwards elimination procedure is run to determine a final model.  
-The final covariate model is estimated. 
-Risk Ratio figures and tables are created if any demographic covariates are included in the final covariate model.
\end{verbatim}

An E-R analysis was performed for all endpoints even if the number of
events was very low.

\#\#Base Model Description In the base model, exposure metrics were
tested as predictors for the occurrence of an event.

The linear, squareroot-transformed, and log-transformed scales of
exposure metrics were investigated. Since the main purpose of the E-R
analysis was to identify potential relationships between endpoints and
exposure metrics, the endpoint was only explored and was not part of the
main analysis conclusions if no exposure metric was defined.

The probability of observing an Adverse Event (AE) is described by the
Poisson distribution: \begin{equation}
\label{eqn:P}
\text{P}\left(Y_{j} = n\right) = \frac{\lambda^{n}}{n!}\cdot e^{-\lambda}
\end{equation} Where \(\text{P}\left(Y_{j} = n\right)\) is the
probability of observing \(Y\) during an interval \(j\), where \(Y\) can
take on values \(n = 0, 1, 2,\dotsc\), \(\lambda\) is an estimable
parameter describing the mean count, and \(!\) is the factorial
function.

The expectation for the Poisson distribution
\(\text{E}\left(Y_{j}\right)\) is equal to \(\lambda\), and therefore
\(\lambda\) is the arithmetic mean of the counts occurring during a
certain time interval. The variance of the Poisson distribution is not
governed by an additional parameter (such as \(\sigma\) for a normal
distribution), but by \(\lambda\) as well. Therefore, \(\lambda\)
defines both the mean and the variance of the distribution of
observations. \begin{equation}
\label{eqn:lambda}
\lambda = \text{E}\left(Y_{j}\right) = \text{Var}\left(Y_{j}\right)
\end{equation}

A base model was developed using a poisson regression with an intercept
and potentially an exposure metric as shown in below. \begin{equation}
\label{eqn:mod}
\text{log}(\lambda) = \beta_{0}+\text{log}\left(t_{j}\right)+\beta_{1}\cdot \text{Exposure}_j
\end{equation}

where \(\lambda\) is the mean count, \(\beta_{0}\) is the mean count
when the time interval, \(t\), is equal to zero,
\(\text{log}\left(t_{j}\right)\) is the time-interval offset (days),
\(\text{Exposure}_j\) is the time-weighted Exposure at the time of AE
driving the response in mean count, \(\beta_{1}\) is the estimable
effect of exposure on the mean count.

When assessing each exposure metric individually in the base model, the
difference in the number of identifiable parameters (and therefore the
\(df\)) is equal to 1. Decision making during model building was guided
by evaluation of change in deviance, or \(-2\cdot\)\{log-likelihood\},
between models. The Deviance (DEV) was calculated as: \newline
\begin{equation}
\label{eqn:dev}
D=-2\log\left(\frac{L_0}{L_F}\right)
\end{equation}

where \(L_0/L_F\) is the ratio of the likelihood of the null and fitted
models. \(D\) can be shown to be approximately \(\chi^2\) distributed
with degrees of freedom equal to the difference in the number of
parameters estimated between the null and fitted models.

Based on user's input, exposure metric selection was done in one of the
following ways:

\begin{verbatim}
-The exposure metric met the siginificant level (p_val) was selected for the base model for each endpoint analyzed. If there were more than one metric that met the criteria, the metric with the smallest $p$-value was selected. If there was no metric that met the criteria, the base model would not contain any exposure metric. 
-The exposure metric with the largest change in deviance $\Delta D$ was selected for the base model for each endpoint analyzed, regardless of statistical significance.
\end{verbatim}

The generated report includes the details of user's choice.

\#\#Random Effects Model Development Since the poission regressions used
only one measurement per subject, estimation of random effects was not
possible.

\#\#Inclusion of Covariates and Full Model Development Both categorical
and continuous covariates were tested for inclusion using a linear
parameterization in the poisson regressions.

The covariate parameter structure that was used for categorical and
continuous covariates is described below:

\begin{verbatim}
-Linear parameterization for a categorical covariate x:
\end{verbatim}

\newline COV = 1; most common \newline
\(\theta=\theta_0 \cdot \text{COV}\) \newline where \(\theta_0\) denotes
the population value of the parameter for the reference demographic
characteristic. The parameter \(\theta\) represents the change in the
intercept when compared to the reference characteristic.

\begin{verbatim}
-Linear parameterization for continuous covariates
\end{verbatim}

\newline

\(\theta=\theta_0 \cdot (\text{COV}-\text{reference})\) \newline where
\(\theta_0\) denotes the population value conditional on the value of
the covariate, which changes linearly as a function of COV, reference
value is usually the observed median value or given by user. Continuous
covariates might be explored in the logarithmic scale and were tested in
the models in the scale where the covariate follows a normal
distribution.

If the user set \texttt{con.model.ref\ =\ "No"} for not using reference
value in continuous covariates parameterization,
\(\text{reference} = 0\).

Categorical covariate would be dropped in full model development if all
events occurred in only one of its categories. If any of the covariates
were highly correlated (e.g.~AST and ALT) with a correlation coefficient
\$\ge\$0.6, then only one of the correlated covariates was tested
further based on univariate DEV. The \(\chi^2\) test for the
log-likelihood difference in deviance between models was used to judge
whether one model had a better fit over another during the backward
elimination using an \(\alpha\) of 0.01. When the removal of any of the
remaining covariates results in a \(\Delta D\) equivalent to p-value
less than 0.01 the elimination process was stopped and the model was
considered final.

\#\#Missing Data Missing data within covariates was imputed provided the
percentage of missing values was \$\le\$10\% for continuous covariates
and \$\le\$10\% for categorical covariates. For continuous covariates,
the median value was used for the imputed value. For categorical
covariates, the mode was used for the imputed value. If the percentage
of missing values was larger than the stated threshold, the covariates
were excluded from consideration for the endpoint.

\#\#Full Model Development To assess the E-R relationship between
potential covariates and each of the endpoints, a poisson regression
model was used as: \begin{equation}
\label{eqn:LRbasecov}
\text{log}(\lambda) = \beta_{0} + \text{log}\left(t_{j}\right) + \beta_1\cdot\text{Exposure}_j + \beta_2\cdot X_2 + \cdots + \beta_n\cdot X_n
\end{equation} where \(\lambda\) is the arithmetic mean of the counts
occurring during a certain time interval, \(\beta_0\) is the estimated
intercept, \(\beta_1\) is a regression coefficient (slope) representing
the effect of the exposure metric, if any, based on base model
development, and \(\beta_2,\ldots,\beta_n\) represent the effect, if
any, of each additional covariate on the log-odds of the event
occurring.

\#\#Final Model Development Final model development started with the
full model, containing the parameters from the base model and any
additional covariates under consideration. The full model was then
subjected to a stepwise backwards elimination procedure, outlined below.

To compare two nested models, the difference in the deviance of each of
the models also follows an approximately \(\chi^2\) distribution with
degrees of freedom equal to the difference in the number of estimated
parameters:

\begin{equation}
\label{eqn:nest}
\left(\frac{D_{nested}-D_{full}}{df_{nested}-df_{full}}\right)\sim\chi^2_{df_{nested}-df_{full}}
\end{equation}

where \(df_{nested}\) and \(df_{full}\) are the degree of freedom for
the nested and full models, respectively.

\end{document}
